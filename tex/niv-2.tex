\section{Level 2}

The chosen representation is as a graph, with \(u \to v\) if and only
if there exists an environment in which \(v\) can be reached from \(u\)
in one execution step (i.e. one assignment or one branching statement).\\

Each statement is represented internally as
\begin{lstlisting}
typedef struct RStep {
    bool advance;
    RAssign* assign;
    unsigned nbguarded;
    struct RGuard* guarded;
    struct RStep* unguarded;
    uint id;
} RStep;
\end{lstlisting}
Where:
\begin{itemize}
    \item \ttt{advance} is used to identify loops
    \item \ttt{assign} represents an assignment
    \item \ttt{guarded} indicates guarded clauses in the event of an
        \ttt{if} or \ttt{do}
    \item \ttt{unguarded} is either an \ttt{else} for an \ttt{if} or \ttt{do}
        clause, or the next step to execute for a normal statement.
\end{itemize}~\\

The algorithm for calculating one step of the execution is as follows:
\begin{enumerate}
    \item if \ttt{assign} is not \ttt{NULL}, perform assignment
    \item if \ttt{nbguarded == 0} then jump to \ttt{unguarded} and terminate
    \item else if one of \ttt{guarded} is satisfied, execute it
    \item otherwise \ttt{unguarded} is an \ttt{else} branch, execute it
\end{enumerate}
If at any point \ttt{NULL} is reached it means the procedure has terminated.\\

Expressions possess direct pointers to the variables they contain, which are
set up during the translation step in \ttt{repr.c}.\\

This level also includes some pretty-printing facilities.

\input{repr.dump}

Note:
\begin{itemize}
    \item variables now have a unique identifier as well, ensuring that
        local variable shadows are implemented correctly
    \item the process indicates its entry point which is where the
        execution will start
    \item each statement has a \ttt{[n]} marker which indicates
        to jump to the corresponding \ttt{<n>}, and \ttt{<n>} is the same
        as in the pretty-printing for \tbf{Level 1}
    \item the continuation of a loop is inlined inside it, and looping
        constructs are identified with \ttt{(loop)}
\end{itemize}~\\


An alternative representation as a graph is also available,\\
for example \ttt{./lang assets/sort.prog \ddash dot \&\& xdg-open assets/sort.prog.png}:

\begin{center}
\includegraphics[height=24cm]{../assets/sort.prog.png}
\end{center}
Conditions are diamonds, assignments are ovals, branching statements
(\ttt{do},\ttt{if}) are triangles, variables are rectangles.

