\section{Usage}

\newcommand{\ddash}{\textminus\textminus}

Flags are available to control the execution of the program.\\

\textbf{Level 1}: \ttt{\ddash ast} (a.k.a. \ttt{-a}) will print the syntax tree in a readable manner\\

\textbf{Level 2}: \ttt{\ddash rand} (\ttt{-R}) will randomly execute 100 steps on 100 instances of the
program\\

\textbf{Level 3}: \ttt{\ddash all} (\ttt{-A}) will exhaustively explore all configurations\\

\textbf{Levels 2\&3}:\\
\ttt{\ddash repr} (\ttt{-r}) will print the internal representation as text,\\
\ttt{\ddash dot} (\ttt{-d}) will dump and render a \ttt{.dot} file for
\href{https://graphviz.org/}{\ttt{graphviz}}\\
\ttt{\ddash trace} (\ttt{-t}) will print for each reachable configuration
a sequence of steps that leads to it being satisfied\\

\textbf{Misc}:\\
\ttt{\ddash help} (\ttt{-h}) will print a help message and exit,\\
\ttt{\ddash no-color} (\ttt{-c}) will turn of ANSI color code formatting for
all pretty-prints\\

\textbf{Examples}:
\begin{lstlisting}
$ make
$ ./lang assets/sort.prog -ar --no-color
# prints the ast and the representation of sort, without special characters
$ ./lang assets/sort.prog --dot
# renders assets/sort.prog.png that represents the program
$ ./lang assets/sort.prog --all --rand -t
# executes sort in both Monte-Carlo and exhaustive modes, prints paths towards
# satisfied conditions
\end{lstlisting}
