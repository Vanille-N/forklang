\section{Other remarks}

\begin{itemize}
    \item I made sure that no memory is leaked by implementing a registry
        that stores a linked list of blocks of memory to free.\\
        More specific information about this is available in the code.\\

        To facilitate the freeing of ressources I decided that only in
        \ttt{main} could \ttt{exit} be called, and all allocations inside
        a function must either be returned by the function, freed before
        the end, or added to one of the registries for \ttt{main} (or the caller) to free.\\
        The only exceptions to this rule are \ttt{HashSet} and \ttt{WorkList},
        since they keep track of all their contents and have their fields hidden.\\

        The \ttt{make valgrind} target will validate the absence of
        possible leaks on a range of valid and invalid inputs.\\

    \item One suggested extension was to display the line number in case
        of a parsing error, so I just turned on a few options to get
        verbose error printing and line numbers.\\

    \item When several variables are declared with the same name, the
        duplicate declarations are ignored. When several procedures are
        declared with the same name it does not affect the computation
        since nothing relies on a procedure's name. It does however impede
        readability of the trace.
\end{itemize}
